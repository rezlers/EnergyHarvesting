
\documentclass[10pt, reqno]{amsart}
\usepackage[utf8]{inputenc}
\usepackage[T2A]{fontenc}
\usepackage[english,russian]{babel}
\usepackage{amsmath}
\usepackage{amssymb}
\usepackage{amsfonts}
\usepackage{graphicx}
\usepackage[hidelinks,unicode]{hyperref}
\usepackage{indentfirst}


\def\udcs{519.233} %Here the author places classificators of the paper according to Russian classification system
\def\mscs{62F03} %Here the author places  classificators according to the AMS classification list.
\setcounter{page}{1}

\newtheorem*{theorem*}{Теорема}
\newtheorem{repeated_theorem}{Теорема}
\newtheorem{numbered_theorem}{Утверждение}
\newtheorem{numbered_corollary}{Замечание}
\newtheorem{lemma}{Лемма}

\newcommand{\norm}[1]{\left\lVert#1\right\rVert}

\DeclareUnicodeCharacter{2212}{-}
\DeclareMathOperator*{\E}{\mathbb{E}}
\DeclareMathOperator*{\Pb}{\mathbb{P}}

\def\logo{{\bf\huge S\raisebox{0.2ex}{\hspace{0.55ex}\raisebox{0.05ex}e\hspace{-1.65ex}$\bigcirc$}MR}}

\makeatletter
\def\@settitle{
    \begin{center}%
    \baselineskip14\p@\relax
    \bfseries
    \large
    \@title
  \end{center}%
}
\makeatother

\def\semrtop
     {
  \vbox{
     \noindent\logo
     \hspace{80mm}\raisebox{1ex}{ISSN 1813-3304 }

     \vspace{5mm}

     \begin{center}
     {\huge СИБИРСКИЕ \ ЭЛЕКТРОННЫЕ} \\[2mm]  %Headings in Russian
     {\huge МАТЕМАТИЧЕСКИЕ ИЗВЕСТИЯ} \\[2mm]
     {\large Siberian Electronic Mathematical Reports} \\[1mm]
     {\LARGE\tt{http://semr.math.nsc.ru}}\\[0.5mm]
%     {\small 3 3 3 3 3}\\[-1mm]
%     {\small Sobolev Institute of Mathematics SB RAS}
     \end{center}
     \vspace{-3mm}
     \noindent
     \begin{tabular}{c}
     \hphantom{aaaaaaaaaaaaaaaaaaaaaaaaaaaaaaaaaaaaaaaaaaaaaaaaaaaaaaaaaaaaaaaaaaaaaa} \\
     \hline\hline
     \end{tabular}

     \vspace{1mm}
     {\flushleft\it Том 16, стр. 144--144 (2019) \hspace{65mm}{\rm\small УДК \udcs}} %Volume, pages in Russian,
                                                                                    %filled by Editorial board
     \newline
         {\rm\small DOI~10.33048/semi.2019.16.xxx}\hphantom{aaaaaaaaaaaaaaaaaaaaaaaaaaaaaaaaaaaa}{\rm\small MSC\ \ \mscs }
  }%\vbox
}

\newcommand\myshorttitle{\small\textsc{Стабильность систем с механизмом энергетической подпитки}}
\begin{titlepage}
\title[\myshorttitle]{Стабильность и нестабильность систем случайного множественного доступа с механизмом энергетической подпитки и эффектом потери энергии}
\author{{А.В. Резлер, М.Г. Чебунин}}%

\address{Alexandr Vadimovich Rezler
\newline\hphantom{iii} Novosibirsk State University
\newline\hphantom{iii} 2, Pirogova str.}%

\email{rezlers123@gmail.com}%
\address{Mikhail Georgievich Chebunin
\newline\hphantom{iii} Karlsruhe Institute of Technology,
\newline\hphantom{iii} Institute of Stochastics,
\newline\hphantom{iii} Karlsruhe, 76131, Germany,
\newline\hphantom{iii} Novosibirsk State University, 
\newline\hphantom{iii} 2, Pirogova str.,
\newline\hphantom{iii} Novosibirsk, 630090, Russia.}%

\email{chebuninmikhail@gmail.com}%



\thanks{\sc A. Rezler, M. Chebunin,
Stability and instability of a random multiple access system with an energy harvesting mechanism}
\thanks{\copyright \ 2021 Резлер А.В., Чебунин М.Г}
\end{titlepage}
\begin{document}
\renewcommand{\refname}{References}
\renewcommand{\proofname}{Доказательство.}
\renewcommand{\figurename}{Fig.}
\thispagestyle{empty}


\semrtop \vspace{1cm}
\maketitle {\small
\begin{quote}
\noindent{\sc Abstract.} 
We study a generalisation of the model of the classical synchronised multiple access system with a single transmission channel controlled by a randomised transmission protocol (ALOHA) and additionally equipped with an energy harvesting mechanism. The generalisation is the assumption that message batteries may receive an unlimited amount of energy. We also consider that each battery is subject to self-discharge, i.e. loss of energy during idle periods.

\medskip

\noindent{\bf Keywords:} Markov chains, ALOHA algorithm, energy harvesting, generalised Foster criterion, ergodicity, transience.
 \end{quote}
}

\section{Введение}
Использование энергетической подпитки (energy harvesting) позволяет устройствам непрерывно получать энергию из различных источников окружающей среды, таких как, например, свет, тепло, вибрация и множество других. Несмотря на то, что количество энергии, получаемое таким образом, сравнительно невелико, данная технология находит широкое применение, например, среди устройств с низким потреблением энергии (Low-Power Systems), потому что снабжение устройств механизмом энергетической подпитки позволяет использовать их в тех места, где классические источники энергии недоступны. Более того, так как заряд батарей непрерывно восполняется, то таким образом может быть продлена продолжительность "жизнь"$ $ используемых устройств. Так, в работе \cite{Harb_Adnan} приведен пример устройства, использующего микроэлектромеханические (MEMS) системы для извлечения энергии из вибраций окружающей среды. Также, механизм энергетической подпитки имеет широкое применение среди систем множественного доступа. Например, сенсорные сети, оснащенные перезаряжающимися батареями, которые "подпитываются"$ $ энергией из окружающей среды, могут существенно продлить срок ее годности (см. \cite{Sudevalayam_Kulkarni}). Другой пример использования данного механизма был предложен в работе Xun Zhou, Rui Zhang и Chin Keong Ho (см. \cite{Example_of_applying_Xun_Zhou}), где авторы рассмотрели многопользовательскую систему с базовой станцией, использующей технологию <<OFDM>> (orthogonal frequency division multiplexing)$ $ и беспроводной канал для передачи информации и энергии пользователям.

Использование механизма энергетической подпитки в системах случайного множественного доступа ставит новые задачи и, в частности, о нахождении пропускных способностей систем. Так, в работе J. Jeon и A. Ephremides (см. \cite{Jeon_finite_users}) рассматривался случай системы с ограниченным количеством пользователей, децентрализованным механизмом энергетической подпитки и протоколом передачи данных типа ALOHA. В работе было предложено выражение для пропускной способности данной системы, а также был продемонстрирован эффект ее уменьшения, при ограничении вместимости у механизма для хранения энергии, которым снабжен каждый пользователь системы. Изучению энергоэффективности протоколов разрешения конфликтов посвящена работа A. Bergman и M. Sidi (см. \cite{Efficiency_protocols}). Как известно, традиционные (в частности, не использующие механизм энергетической подпитки) системы множественного доступа (см. например \cite{Abramson_dev_of_aloha_net}, \cite{FGL_1977}) с бесконечным числом пользователей в известном смысле нестабильны ни при каких значениях управляемых параметров (см. \cite{Kelly_classical_aloha_instability}) и метастабильны при малых значениях входного потока и вероятности передачи сообщений (см. \cite{Vvedenskaya_metastability}). Однако, как было показано в работах С.Г. Фосса, Д.К. Кима и А.М. Тюрликова \cite{Foss_2014} и \cite{Foss_original}, ограничения на поступление энергии и входной поток в систему с бесконечным числом пользователей может стабилизировать ее, то есть количество сообщений в накопителе системы не будет неограниченно расти. В работе \cite{Foss_2014} были найдены области стабильности и нестабильности модели для классической синхронизированной системы случайного множественного доступа с одним передающим прибором, управляемой протоколом передачи типа ALOHA, снабженной централизированным механизмом энергетической подпитки. Однако, как отмечают авторы, рассмотренная модель далека от практики. Тем не менее, рассмотренная в работе \cite{Foss_original} модель для децентрализованного механизма энергетической подпитки, в которой каждый пользователь системы снабжен индивидуальным механизмом для хранения энергии, имеет практические аналоги. В упомянутой работе, при определении конкретного вида функции вероятности подзарядки сообщений на каждом шаге, обратно пропорциональной общему количеству сообщений, присутствующих в системе, была найдена пропускная способность модели. В нашей работе 2022г. \cite{Rezler_Chebunin} изучается обобщение предыдущей модели. Главное отличие состоит в том, что теперь сообщения могут принимать более одной единицы энергии. Одной из причин изучения обобщенной модели являлся эффект, продемонстрированный в упомянутой работе \cite{Jeon_finite_users}, возникающий при увеличении емкости накопителя энергии у пользователей и, как следствие, приводящий к расширению областей стабильности системы. Однако результаты, полученные в нашей статье, показывают, что максимальная пропускная способность системы, при заданной интенсивности подзарядки, не меняется. 

Однако, любая система, использующая механизм энергетической подпитки, помимо "сборщика"$ $ энергии, снабжена аккумулятором, который, как известно, имеет свойство саморазрядки, то есть потери энергии в период простоя. Таким образом, представляет интерес учитывать данное свойство в моделях, соответствующих описываемым системам. В данной работе мы обобщаем результаты, полученные в нашей работы 2022г. \cite{Rezler_Chebunin}, на случай, учитывающий эффект саморазрядки.

Работа состоит из 5 параграфов, заключения и приложения. Во втором параграфе описывается математическая модель и формулируется основная теорема, в третьем параграфе проводится доказательство для частного случая описанной модели, когда каждая батарейка имеет только одну ячейку для хранения энергии. В четвертом параграфе проводится доказательство основной теоремы. В пятом параграфе рассматривается случай, когда каждая батарейка имеет произвольное ограниченное количество ячеек для хранения энергии. В приложении приводятся формулировки утверждений, используемых при доказательстве основной теоремы.

\section{Описание моделей и основные результаты}
Мы рассматриваем систему с одним передающим прибором, которая принимает и, при успешной передаче, отправляет сообщения. Время слотировано. Пусть $\xi_{n}$ --- случайная величина, определяющая количество сообщений, поступивших в накопитель системы в течении интервала времени $[n-1, n)$. Далее предполагаем, что $\{\xi_{n}\}_{n \geq 0}$ образует последовательность независимых одинаково распределенных (н.о.р.) неотрицательных целочисленных случайных величин с конечным математическим ожиданием $\lambda$. Каждое сообщение снабжено батареей с неограниченным количеством ячеек для хранения энергии и прибывает в накопитель системы с пустой батареей, ожидая энергетической подпитки. Полный цикл работы системы в каждый временной слот опишем далее. Каждое сообщение, имеющее в начале временного слота батарею, заряженную на $i$ ячеек, где $i \geq 1$, будет передано на передающий прибор с вероятностью $1-p^{i}$ или останется в системе с вероятностью $p^{i}$, где $p \in (0, 1)$ фиксировано ($p=0$ мы не рассматриваем, так как система, в данном случае, функционирует в соответствии со стандартным центрированным протоколом ALOHA). После передачи, каждое заряженное сообщение, не пытавшееся передаться на передающий прибор, независимо от остальных параметров системы, потеряет одну единицу энергии с вероятностью $\hat{p}$. Далее каждое сообщение (помимо тех сообщений, которые в данном временном слоте потеряли единицу энергии), независимо от остальных параметров системы, получит одну единицу энергии с вероятностью $\mu > 0$. Затем, если в заданный временной интервал на передающий прибор поступило только одно сообщение, то оно покидает систему. Если на передающий прибор поступило два или более сообщений, то происходит наложение (конфликт) и передававшиеся сообщения возвращаются в систему, но теряют одну единицу энергии.


Теперь предположим, что к началу, скажем, $(n + 1)$-го временного слота, $\xi_{n}$ --- новые сообщения, которые только прибыли в накопитель системы, а $q_{n}$ --- сообщения, которые уже находились в накопителе системы к началу данного временного слота. Пусть $v^{(i)}_{n}$ --- количество сообщений, имеющих $i$ единиц энергии, где $i \geq 1$, к началу данного временного слота. Очевидно, что
\begin{align*}
    \sum_{i=1}^{\infty}v^{(i)}_{n} \leq q_{n}, \text{ п.н.}
\end{align*}
Обозначим через $\{U^{(j)}_{n,i}\:, \:-\infty < n < \infty,\: i \geq 1,\: 1 \leq j \leq \infty\}$, $\{\widetilde{U}^{(j)}_{n,i}\:,\: -\infty < n < \infty,\: i \geq 1,\: 1 \leq j \leq \infty\}$ и $\{\dot{U}^{(j)}_{n,i}\:,\: -\infty < n < \infty,\: i \geq 1,\: 1 \leq j \leq \infty\}$ независимые семейства н.о.р. случайных величин, имеющих равномерное распределение на отрезке $[0, 1]$. 

Пусть $I(A)$ --- индикаторная функция события $A$: она принимает значение, равное 1, если событие происходит, и 0 иначе. Введем также некоторую измеримую неотрицательную функцию $\mu:\mathbb{Z}_{+} \to [0,1]$, которая будет подчеркивать свойство адаптивности модели. То есть в зависимости от количества сообщений, скажем, в $n$-ый момент времени $q_{n}$, значение $\mu(q_{n})$ будет равно вероятности подзарядки каждого сообщения в $n$-ый момент времени. Теперь пусть $B^{(j)}_{n}(k, 1-p^{j}) = \sum_{i=1}^{k}I(U^{(j)}_{n,i} < 1-p^{j})$, $\widetilde{B}^{(j)}_{n}(k, \mu) = \sum_{i=1}^{k}I(\widetilde{U}^{(j)}_{n,i} < \mu)$ и $\dot{B}^{(j)}_{n}(k, \mu) = \sum_{i=1}^{k}I(\dot{U}^{(j)}_{n,i} < \mu)$, где  $-\infty < n < \infty,\: k \geq 0,\: j \geq 1$, три взаимно независимых семейства случайных величин, имеющих биномиальное распределение и независимых от остальных параметров системы. Обозначим также через $D^{(j)}_{n}(k, p^{j}) = k - B^{(j)}_{n}(k, 1-p^{j})$. Очевидно, что случайная величина $D^{(j)}_{n}(k, p^{j})$ имеет биномиальное распределение с параметрами $p^{j}$, $k$. Аналогично введем случайную величину $\dot{D}^{(j)}_{n}(k, 1 - \hat{p})$. Пусть $I_{p}(n) = I(\sum_{i=1}^{\infty}B_{n}^{(i)}(v_{n}^{(i)}, 1-p^{i}) = 1)$, то есть $I_{p}(n)$ --- индикатор события, состоящего в том, что в $n$-ый момент времени произошла успешная передача сообщения. Таким образом, мы получим следующую рекурсию.

\small
\begin{equation}
\begin{cases}
q_{n+1} = q_{n} − I_{p}(n) + \xi_{n}, \\\\
v_{n+1}^{(1)} = \dot{D}_{n}^{(1)}(D_{n}^{(1)}(v_{n}^{(1)}, p), 1-\hat{p}) + \widetilde{B}_{n}^{(1)}(q_{n} − \sum_{i=1}^{\infty}v_{n}^{(i)} + \xi_{n}, \mu(q_{n})) \\- \widetilde{B}_{n}^{(2)}(\dot{D}_{n}^{(1)}(D_{n}^{(1)}(v_{n}^{(1)}, p), 1-\hat{p}), \mu(q_{n})) + \dot{B}_{n}^{(2)}(D_{n}^{(2)}(v_{n}^{(2)}, p^{2}), \hat{p}) \\+ B_{n}^{(2)}(v_{n}^{(2)}, 1-p^{2})\cdot(1-I_{p}(n)) \\\\
v_{n+1}^{(2)} = \dot{D}_{n}^{(2)}(D_{n}^{(2)}(v_{n}^{(2)}, p^{2}), 1-\hat{p}) + \widetilde{B}_{n}^{(2)}(\dot{D}_{n}^{(1)}(D_{n}^{(1)}(v_{n}^{(1)}, p), 1-\hat{p}), \mu(q_{n}))\\- \widetilde{B}_{n}^{(3)}(\dot{D}_{n}^{(2)}(D_{n}^{(2)}(v_{n}^{(2)}, 1-p^{2}), 1-\hat{p}), \mu(q_{n})) + \dot{B}_{n}^{(3)}(D_{n}^{(3)}(v_{n}^{(3)}, p^{3}), \hat{p}) \\+ B_{n}^{(3)}(v_{n}^{(3)}, 1-p^{3})\cdot(1-I_{p}(n))\\
...
\end{cases}
\label{not_coupling_equation}
\end{equation}
\normalsize

Для доказательства необходимых свойств системы нам достаточно рассмотреть модель, в которой вместо двух независимых семейств случайных величин $\{U^{(j)}_{n,i}\}$ и $\{\dot{U}^{(j)}_{n,i}\}$ мы будем рассматривать семейство независимых равномерно распределенных случайных величин $\{U^{(j)}_{n,i}\}$, принимающих значения в $[0, 1]^{2}$. Другими словами, для исследования необходимых нам свойств, мы можем отождествлять потерю заряда при коллизии и при саморазрядке. Тогда случайные величины $D^{(j)}_{n}(k, p^{j}(1-\hat{p}))$ будут определяться следующим образом:

\begin{align*}
    D^{(j)}_{n}(k, p^{j}(1-\hat{p})) = \sum_{i=1}^{k}I(\:U^{(j)}_{n,i} \in [0,\: p^{j}] \times [0,\: (1-\hat{p})]\:)
\end{align*}

Ясно, что $D^{(j)}_{n}(k, p^{j}(1-\hat{p}))$ имеют биномиальное распределение с параметрами $k$ и $p^{j}(1-\hat{p})$. Введем также биномиальные случайные величины $B^{(j)}_{n}(k, 1 - p^{j}(1-\hat{p}))$ с соответствующими параметрами. Таким образом мы можем записать модель (\ref{not_coupling_equation}) в следующем виде:

\small
\begin{equation*}
\begin{cases}
q_{n+1} = q_{n} − I_{p}(n) + \xi_{n}, \\\\
v_{n+1}^{(1)} = D_{n}^{(1)}(v_{n}^{(1)}, p(1-\hat{p})) + \widetilde{B}_{n}^{(1)}(q_{n} − \sum_{i=1}^{\infty}v_{n}^{(i)} + \xi_{n}, \mu(q_{n})) \\- \widetilde{B}_{n}^{(2)}(D_{n}^{(1)}(v_{n}^{(1)}, p(1-\hat{p})), \mu(q_{n})) + D_{n}^{(2)}(v_{n}^{(2)}, p^{2}\hat{p}) \\+ B_{n}^{(2)}(v_{n}^{(2)}, 1-p^{2})\cdot(1-I_{p}(n))\\\\
v_{n+1}^{(2)} = D_{n}^{(2)}(v_{n}^{(2)}, p^{2}(1-\hat{p})) + \widetilde{B}_{n}^{(2)}(D_{n}^{(1)}(v_{n}^{(1)}, p(1-\hat{p}))), \mu(q_{n})) \\- \widetilde{B}_{n}^{(3)}(D_{n}^{(2)}(v_{n}^{(2)}, p^{2}(1-\hat{p})), \mu(q_{n})) + D_{n}^{(3)}(v_{n}^{(3)}, p^{3}\hat{p}) \\+ B_{n}^{(3)}(v_{n}^{(3)}, 1-p^{3})\cdot(1-I_{p}(n))\\
...
\end{cases}
\tag{M2}
\label{Ф2}
\end{equation*}
\normalsize

Заметим, что стохастическая последовательность $(q_{n}, v^{(1)}_{n}, v^{(2)}_{n}...)$, образует бесконечномерную марковскую цепь. При том пространством состояний является
\begin{align}
    \mathfrak{X} = \{(q_{n}, v^{(1)}_{n}, v^{(2)}_{n}...) \in \mathbb{Z}_{+}^{\infty} \: : \: \sum_{i=1}^{\infty}v^{(i)}_{n} \leq q_{n}, \: n \geq 0\}.
    \label{State_space}
\end{align}
Ради краткости написания, далее будем отождествлять модели и соответствующие им стохастические последовательности. Будем говорить, что система работает стабильно (для краткости --- цепь стабильна), если соответсвующая ей цепь Маркова сходится к своему стационарному распределению в метрике полной вариации из любого начального состояния.

Построенная модель (\ref{Ф2}) предполагает наличие у каждого пользователя механизма для хранения неограниченного числа ячеек энергии. В существующих системах с передачей энергии по радиоканалу (см., например, \cite{Example_of_applying_Xun_Zhou}) энергия, передаваемая базовой станцией, накапливается у каждого абонента в своем индивидуальном источнике и далее используется для передачи информации. Однако, каждый индивидуальный источник имеет только одну ячейку для хранения энергии. Таким образом, модель (\ref{Ф2}) отражают работу беспроводных систем, снабженных механизмом энергетической подпитки, с тем отличием от существующих, что в модели дополнительно предполагается наличие механизма для хранения неограниченного количества ячеек энергии у пользователей сети. 

Сформулируем основные результаты данной работы. Сначала предположим, что $\lambda < 1$. Как будет видно ниже, данное допущение не ограничивает общность результатов. Следовательно $\Pb(\xi_{1} = 0) > 0$. То есть цепь Маркова (\ref{Ф2}) за один шаг с положительной вероятностью не изменит состояние. Следовательно, рассматриваемая цепь является апериодичной. Более того, можно заметить, что на каждом шаге с положительной вероятностью из системы уходят и в нее поступают сообщения, а также в системах с положительной вероятностью сообщения <<подзаряжаются>> на произвольное количество ячеек энергии за некоторое количество шагов. Таким образом, можно утверждать, что все их состояния являются сообщающимися. Следовательно, из эргодической теоремы, для определения пропускной способности системы, достаточно найти компактное положительно возвратное множество у полученных цепей, чему и посвящена большая часть данной работы.

Перейдем к рассмотрению основной теоремы, доказательство которой основано на утверждениях, сформулированных в Приложении (см. работы \cite{Foss_original}, \cite{Foss_theorem_for_instability} и \cite{Foss_Konstantinopolous}). Также всюду далее мы будем использовать следующие обозначения:
\begin{gather*}
    x = (x_{1}, x_{2},...),\\
    \E{}_{x}(\cdot) = E(\cdot \:|\: X_{0} = x),\\
    \Pb{}_{x}(\cdot) = \Pb(\cdot \:|\: X_{0} = x),\\
    \norm{x}_{l_{1}} = \norm{x} = \sum_{i=1}^{\infty}x_{i}.
\end{gather*}

\begin{theorem*}
Если, для некоторой константы $c > 0$, $\mu(q)$ имеет вид
\begin{align*}
    \mu(q) = \begin{cases}
    \min(\widetilde{c}/q, 1),\:\:q \in \mathbb{Z}_{+}\backslash\{0\}\\
    1,\:\: q = 0
    \end{cases},
\end{align*}
где $\widetilde{c} = (1 - p(1-\hat{p}))(1-p)^{-1}c$, тогда стохастическая последовательность, представляемая бесконечномерной марковской цепью (\ref{Ф2}) стабильна при входной интенсивности $\lambda < ce^{−c}$ и нестабильна при $\lambda > ce^{−c}$.
\end{theorem*}
Заметим, что, при $c = (1 - p(1-\hat{p}))^{-1}(1-p)$, мы получим максимально допустимую область стабильности системы $(\lambda < e^{-1})$.

\section{Случай единичной вместимости}
В работе С.Г. Фосса и соавторов \cite{Foss_original} было показано, что если для некоторой константы $c > 0$
\begin{equation}
    \mu(q) = \begin{cases}
    \min(c/q, 1),\:\:q \in \mathbb{Z}_{+}\backslash\{0\}\\
    1,\:\: q = 0
    \end{cases} 
    \label{mu_definition}
\end{equation}
то двумерная цепь Маркова, описывающая систему, в которой у каждого сообщения аккумулятор имеет только одну ячейку для хранения энергии
\begin{align}
\begin{cases}
    q_{n+1} = q_{n} - I(B_{n}(v_{n}, 1-p) = 1) + \xi_{n}, \\
    v_{n+1} = D_{n}(v_{n}, p) + \widetilde{B}_{n}(q_{n} - v_{n} + \xi_{n}, \mu(q_{n}))
\tag{M0'}
\label{Ф0'}
\end{cases}
\end{align}
стабильна при входной интенсивности $\lambda < ce^{−c}$ и нестабильна при $\lambda > ce^{−c}$.\\

В модели (\ref{Ф2}) положим $v^{(i)}_{n} \equiv 0$ для всех $i \geq 2$, $n \geq 0$, тогда получим
\begin{align}
\begin{cases}
    q_{n+1} = q_{n} − I(B_{n}(v^{(1)}_{n}, 1-p) = 1) + \xi_{n}, \\
    v^{(1)}_{n+1} = D^{(1)}_{n}(v^{(1)}_{n}, p(1-\hat{p})) + \widetilde{B}_{n}(q_{n} − v^{(1)}_{n} + \xi_{n}, \mu(q_{n}))
\end{cases}
\tag{M0}
\label{Ф0}
\end{align}

Заметим, что вышеупомянутые двумерные цепи Маркова отличаются только в уравнениях для компонент $v_{n+1}$ и $v^{(1)}_{n+1}$ первым слагаемым, определяющим количество заряженных сообщений, не участвовавших в передаче. Легко понять, что стартуя из одного начального состояния, в конце первого временного интервала у цепи (\ref{Ф0}) будет меньше заряженных сообщений, чем у цепи (\ref{Ф0'}), в силу наличия у первой эффекта саморазрядки, однако в остальном, в течении первого временного интервала, системы будут вести себя одинаково. Таким образом, используя естественный каплинг, получим следующие соотношения
\begin{align}
    v^{(1)}_{1}(\ref{Ф0}) \leq v_{1}(\ref{Ф0'}) \text{, } q_{1}(\ref{Ф0}) = q_{1}(\ref{Ф0'}) \text{ п.н.}\notag\\
    B^{(1)}_{n}(v^{(1)}_{n}, 1-p)(\ref{Ф0}) = B_{n}(v_{n}, 1-p)(\ref{Ф0'})  \text{ п.н.}
    \label{coupling_connection}
\end{align}

В работе \cite{Foss_original} для доказательства стабильности цепи (\ref{Ф0'}) используется обобщенный критерий Фостера (см. Приложения, утверждение 2). А именно, сперва авторы выбирают пробную функцию $L((q, v)) = q + v$. Функцию $g((q, v))$ подбирают в виде ступенчатой
\begin{align}
    g((q, v)) = \begin{cases}
    1,\: v > R,\\
    k,\: v \leq R,
    \end{cases}
    \label{step_function_g_stab}
\end{align}
где значение $R > 0$ такое, что при $v > R$ <<снос>> цепи отрицателен, то есть $\E(L(X_{1}) − L(X_{0}) | X_{0}=x) < 0$, где $X_{n} = (q_{n}, v_{n})$, а значение $k \in \mathbb{Z}_{+}$ выбирают таким, чтобы, при достаточно большом $q > \widetilde{q}$, снос цепи также становился отрицательным. Таким образом, внутри компактного множества $[0, \widetilde{q}] \times [0, R]$ снос цепи ограничен, а вне него -- отрицателен.

Для доказательства нестабильности цепи (\ref{Ф0'}) в работе \cite{Foss_original} используется схожий метод (см. Приложения, утверждение 3), а именно, сперва авторы выбирают пробную функцию Ляпунова $L((q, v)) = q$. Функцию $g((q, v))$ подбирают, также, в виде ступенчатой
\begin{align}
    g((q, v)) = \begin{cases}
    1,\: v > \widetilde{R},\\
    k,\: v \leq \widetilde{R},
    \end{cases}
    \label{step_function_g_instab}
\end{align}
где значение $\widetilde{R} > 0$ такое, что, при $v > \widetilde{R}$ и произвольном значении $q_{0}$, <<снос>> цепи положителен, то есть $\E(L(X_{1}) − L(X_{0}) | X_{0}=x) > 0$, а значение $k \in \mathbb{Z}_{+}$ выбирают таким, чтобы, при $v \leq \widetilde{R}$ и достаточно большом $q > \widetilde{q}$, снос цепи также становился положительным. Таким образом, вне множества $\{(q, v) \: | \: q \geq \widetilde{q}, v \leq \widetilde{R}\}$, снос цепи (\ref{Ф0'}) положителен.

Докажем следующую теорему для цепи (\ref{Ф0})
\begin{theorem*}
Если, для некоторой константы $c > 0$, $\mu(q)$ имеет вид
\begin{align*}
    \mu(q) = \begin{cases}
    \min(\widetilde{c}/q, 1),\:\:q \in \mathbb{Z}_{+}\backslash\{0\}\\
    1,\:\: q = 0
    \end{cases},
\end{align*}
где $\widetilde{c} = (1 - p(1-\hat{p}))(1-p)^{-1}c$, тогда стохастическая последовательность, представляемая бесконечномерной марковской цепью (\ref{Ф0}) стабильна при входной интенсивности $\lambda < ce^{−c}$ и нестабильна при $\lambda > ce^{−c}$.
\end{theorem*}
\begin{proof}
Начнем с доказательства стабильности цепи. Рассуждения проведем аналогично доказательству стабильности цепи (\ref{Ф0'}), а именно выберем ту же пробную функцию $L((q, v)) = q + v$, а также будем искать $g((q, v))$ в виде (\ref{step_function_g_stab}). Учитывая соотношения (\ref{coupling_connection}), можно заметить, что $R(\ref{Ф0}) \leq R(\ref{Ф0'})$. Таким образом, для доказательства стабильности, достаточно найти подходящее значение $k \in \mathbb{Z}_{+}$. Проведем рассуждение в несколько этапов:\\
а) Покажем, что последовательность $\{v_{n}\}_{n \geq 0}$ мажорируется цепью, имеющей ограниченный снос.\\
б) Покажем, что последовательность $\{v_{n}\}_{n \geq 0}$, при достаточно большом значении $q_{0}$, близка к некоторой вспомогательной цепи, сходящейся в метрике полной вариации к пуассоновскому распределению.\\
в) Подберем $k$ так, чтобы при $g((q,v)) = k$ и достаточно большом значении $q_{0}$, снос цепи стал отрицательным.\\

а) Пусть $\{Y_{n}\}_{n \geq 0}$ последовательность н.о.р. случайных величин, которые не зависят от ${\xi_{n}}$ и имеют следующее распределение
\begin{equation}
    \Pb(Y_{n} > x) = \underset{q \geq c}{\rm sup}\Pb(B_{1}(q, \mu(q)) > x).
    \label{Y_definition}
\end{equation}
Используя неравенство Маркова, можно заключить, что для $x > 0$ и для любого $\alpha > 0$, правая часть равенства (\ref{Y_definition}) не превосходит
\begin{align*}
    \underset{q \geq c}{\rm sup}\E\frac{e^{\alpha B_{1}(q, c/q)}}{e^{\alpha x}} = \underset{q \geq c}{\rm sup}\Big(1 + (e^{\alpha} - 1)\frac{c}{q}\Big)^{q}e^{-\alpha x} \equiv C_{1}e^{-\alpha x},
\end{align*}
где $C_{1}$ конечна, так как $\Big(1 + (e^{\alpha} - 1)\frac{c}{q}\Big)^{q} \xrightarrow{} \exp(c(e^{\alpha}-1)) < \infty$, при $q \xrightarrow{} \infty$.
\\Таким образом, распределение $Y_{1}$ --- собственное, то есть $\Pb(Y_{n} < \infty) = 1$ для всех $n \geq 0$ и, более того, имеет конечный экспоненциальный момент.

Теперь рассмотрим вспомогательную последовательность
\begin{align}
    Z_{n}^{(1)} = Y_{n}^{(1)} + \xi_{n}, \:\: W_{n+1}^{(1)} = D_{n}^{(1)}(W_{n}^{(1)}, p(1-\hat{p})) + Z_{n}^{(1)},\:\:  W_{0}^{(1)}=v_{0}^{(1)}.
    \label{auxilarry_chain_for_bounds}
\end{align}
Тогда, из свойств (IV) и (V) утверждения 1 (см. Приложения)

\begin{equation}
    \E(v_{n}^{(1)}\:|\:v_{0}^{(1)}) \leq \E(W_{n}^{(1)}\:|\:W_{0}^{(1)}) \leq v_{0}^{(1)} + C^{(1)},
    \label{v_estimation}
\end{equation}
где\\
\begin{equation*}
    C^{(1)} = \frac{E(Y_{1} + \xi_{1})}{1-p(1-\hat{p})} = \frac{EY_{1} + \lambda}{1-p(1-\hat{p})}.
\end{equation*}
\\

б) Рассмотрим вспомогательную последовательность 
\begin{equation}
    \widetilde{V}_{n+1} = D_{n}(\widetilde{V}_{n}, p(1-\hat{p})) + \eta_{n},
    \label{auxiliary_chain}
\end{equation}
где $\{\eta_{n}\}_{n \geq 0}$ --- семейство н.о.р. случайных величин с распределением Пуассона с параметром $\widetilde{c}$ ($\Pi_{\widetilde{c}}$). Используя (III) и (VI) пункты утверждения 1 (см. Приложения), можем заключить, что последовательность $\widetilde{V}_{n}$ сходится к $\pi$ --- распределению Пуассона с параметром $\widetilde{c}/(1-p(1-\hat{p})) = c/(1-p)$ в метрике полной вариации. Тогда, по определению, мы можем выбрать $l \geq 1$ так, что для всех $n \geq l$ 
\begin{equation}
    \underset{\rm \widetilde{V}_{0} \leq R}{\rm sup}|\Pb{}_{x}(B^{(1)}_{n}(\widetilde{V}_{n}, 1-p) = 1) − ce^{-c}| < \delta/2.
    \label{poison_theorem_metrics}
\end{equation}
Очевидно, что для любого значения $k$ можно выбрать константу $C$ так, что вероятность события
\begin{equation*}
    A_{1}(k, C) = \{W_{n}^{(1)} \leq C \:\: \text{для всех} \:\: 0 \leq n \leq k\}
\end{equation*}
будет не меньше, чем $1 - \delta/2$.
Теперь выберем $\widetilde{q}$ так, что расстояние между $\widetilde{B}^{(1)}_{n}(q_{n} - v_{n}^{(1)} + \xi_{n}, \mu(q_{n}))$ для любого фиксированного $n$, при $q_{n} \geq \widetilde{q}$ и распределением Пуассона с параметром $\widetilde{c}$ не превосходит $\delta/2$ в метрике полной вариации равномерно по $v_{n}^{(1)} \in (0, 1, ..., C)$, что можно сделать по известной теореме Пуассона. Более того, мы можем считать, что
\begin{align*}
    \Pb{}_{x}(\{\widetilde{B}^{(1)}_{n}(q - v^{(1)} + \xi_{n}, \mu(q)) \neq \eta_{n}, \text{ при } 0 \leq n \leq k\}, A_{1}(k, C)) < \delta/2
\end{align*}
Откуда следует
\begin{align}
    \Pb{}_{x}(\{v_{n}^{(1)} \neq \widetilde{V}_{n}, \text{ при } 0 \leq n \leq k\}, A_{1}(k, C)) < \delta/2
    \label{charged_poissonised}
\end{align}
\\

в) Выберем $k > l$ такое, что
\begin{align}
    −\varepsilon = k\lambda + C^{(1)} − (k−l)(ce^{-c}−\delta) \notag\\= k(\lambda - (ce^{-c} - \delta)) + l(ce^{-c} - \delta) + C^{(1)} < 0.
    \label{stability_epsilon_define}
\end{align}
Также заметим, что, в силу (\ref{poison_theorem_metrics}) и (\ref{charged_poissonised}), при всех $l \leq n \leq k$, справедливо неравенство
\begin{align}
    \underset{\rm x \leq R}{\rm sup}|\Pb{}_{x}(B^{(1)}_{n}(v^{(1)}_{n}, 1-p) = 1) - ce^{-c}| < \delta
    \label{succes_transmission_poisson}
\end{align}
Наконец, из (\ref{stability_epsilon_define}) и (\ref{succes_transmission_poisson}), получим
\begin{align*}
    \E{}_{x}(L(X_{k}) − L(x)) = \E{}_{x}(q_{k} - q_{0}) + \E{}_{x}(v_{k} - v_{0}) \notag \\\leq  k\lambda - \sum_{i=l+1}^{k}\Pb{}_{x}(I_{p}(n) = 1) + C^{(1)} < k\lambda - \sum_{i=l+1}^{k}(ce^{-c} - \delta) + C^{(1)} \\= k\lambda - (k - l)(ce^{-c} - \delta) + C^{(1)} < -\varepsilon,
\end{align*}
\\

Перейдем к доказательству нестабильности цепи (\ref{Ф0}). Мы будем использовать утверждение 3, сформулированное в приложении. Пусть $n_{k}$ --- некоторая последовательность натуральных чисел такая, что $n_{k} \xrightarrow{} \infty$, при $k \xrightarrow{} \infty$, а также $|n_{k+1}-n_{k}| \leq C < \infty$ для любого $k \geq 0$. В силу неравенства
\begin{align}
    q_{n} \geq q_{n_{k}} - C \:\: \text{п.н. для всех} \:\: n_{k} \leq n  \leq n_{k+1} \:\:\text{и}\:\: k \geq 0,
    \label{q_subseq_inequality}
\end{align}
можно утверждать, что, если $q_{n_{k}} \xrightarrow{} \infty$, при $k \xrightarrow{} \infty$, то $q_{n} \xrightarrow{} \infty$, при $n \xrightarrow{} \infty$. Таким образом, если положить $\widetilde{L}((q, v)) = q$ и
\begin{align*}
    \widetilde{g}((q, v)) = \begin{cases}
    1,\:\: v > \widetilde{R}\:\:\\
    \widetilde{k},\:\: v \leq \widetilde{R}\:\:
    \end{cases},
\end{align*}
где константы $\widetilde{k}$, $\widetilde{R}$ будут заданы далее подходящим образом, то при заданных в условии утверждения 3 ограничениях, достаточно показать, что, во-первых, марковская цепь имеет положительный снос
\begin{equation}
    \E{}_{x}(\Delta_{\widetilde{g}(x)}I(\Delta_{\widetilde{g}(x)} \leq M)) \geq \widetilde{\varepsilon},
    \label{instab_cond_1}
\end{equation}
где $\Delta_{\widetilde{g}(x)} = \widetilde{L}(X_{\widetilde{g}(x)}) - \widetilde{L}(x)$. И, во-вторых, что следующее семейство случайных величин
\begin{equation}
    (\Delta^{-}_{\widetilde{g}(x)})^{2} = (\min(0, \Delta_{\widetilde{g}(x)}))^{2}
    \label{instab_cond_2}
\end{equation}
равномерно интегрируемо. Тогда из условий (\ref{q_subseq_inequality}), (\ref{instab_cond_1}) и (\ref{instab_cond_2}), из соответствующего утверждения, будет следовать
\begin{equation*}
    \Pb{}_{x}(\widetilde{L}(X_{n}) \xrightarrow{} \infty) = 1.
\end{equation*}
В силу неравенства (\ref{q_subseq_inequality}), можно утверждать, что, при выбранных функциях $\widetilde{L}$ и $\widetilde{g}$, верно $0 \geq \Delta^{-}_{\widetilde{g}(x)} \geq -\widetilde{k}$ п.н. и, таким образом, условие (\ref{instab_cond_2}) выполнено. Также, в силу того, что $\E\xi_{1} = \lambda < \infty$, условие (\ref{instab_cond_1}) эквивалентно
\begin{equation}
    \E{}_{x}(\Delta_{\widetilde{g}(x)}) \geq \varepsilon',
    \label{instab_cond_3}
\end{equation}
для некоторого $\varepsilon' > 0$. Следовательно, для доказательства нестабильности, нужно показать, что выполняется неравенство (\ref{instab_cond_3}).

Дальнейшие рассуждения проведем, также, аналогично доказательству нестабильности цепи (\ref{Ф0'}). В силу (\ref{coupling_connection}) и рассуждений проведенных ниже для случая нестабильности цепи (\ref{Ф0'}), можно утверждать, что
\begin{align*}
    \widetilde{R}(\ref{Ф0}) = R(\ref{Ф0'})
\end{align*}

Таким образом, при $v^{(1)}_{0} > \widetilde{R}$
\begin{equation*}
    \E{}_{x}(q_{1} - q_{0}) > 0.
\end{equation*}
Чтобы завершить доказательство нестабильности цепи (\ref{Ф0}), достаточно выбрать некоторый момент времени $k$, когда снос цепи становится положительным. Для этого сначала заметим, что из рассуждений, проведенных при доказательстве стабильности цепи, можно заключить, что справедливо соотношение (\ref{succes_transmission_poisson}). Таким образом, у нас, в том числе, есть некоторый момент времени $l$, когда вспомогательная цепь достаточно близка к своему стационарному распределению. Выберем значение $\widetilde{k}$ так, что выполняется следующее неравенство
\begin{align}
    \varepsilon' = \widetilde{k}\lambda − l -  (\widetilde{k}−l)(ce^{-c} + \delta) \notag\\= \widetilde{k}(\lambda - (ce^{-c} + \delta)) + l)(ce^{-c} - \delta) - l > 0,
    \label{instability_epsilon_define}
\end{align}
где $\delta$ такая, что
\begin{align*}
    0 < \delta < \lambda - ce^{-c}.
\end{align*}

Таким образом, при $q_{0} > \widetilde{q} + \widetilde{k}$ и $v^{(1)}_{0} \leq \widetilde{R}$, получим
\begin{align*}
    \E{}_{x}(\widetilde{L}(X_{\widetilde{k}}) − \widetilde{L}(x)) \geq \widetilde{k}\lambda - l - \sum_{i=l+1}^{\widetilde{k}}\Pb{}_{x}(B_{i}^{(1)}(v_{i}^{(1)}, 1-p) = 1) \notag\\ > \widetilde{k}\lambda - l - \sum_{i=l+1}^{\widetilde{k}}(ce^{-c} + \delta) \notag\\ = \widetilde{k}\lambda - l - (\widetilde{k} - l)(ce^{-c} + \delta) > \varepsilon',
\end{align*}
где $\varepsilon'$ определена в (\ref{instability_epsilon_define}). Доказательство нестабильности цепи (\ref{Ф0}) завершено.
\end{proof}
\section{Случай неограниченной вместимости}
Для доказательства теоремы мы будем использовать результаты и рассуждения из работы \cite{Rezler_Chebunin}. Мы рассматривали следующую бесконечномерную марковскую цепь
\small
\begin{align*}
\begin{cases}
q_{n+1} = q_{n} − I_{p}(n) + \xi_{n}, \\\\
v_{n+1}^{(1)} = v_{n}^{(1)} − B_{n}^{(1)}(v_{n}^{(1)}, 1-p) + \widetilde{B}_{n}^{(1)}(q_{n} − \sum_{i=1}^{\infty}v_{n}^{(i)} + \xi_{n}, \mu(q_{n})) \\- \widetilde{B}_{n}^{(2)}(v_{n}^{(1)} - B_{n}^{(1)}(v_{n}^{(1)}, 1-p), \mu(q_{n})) + B_{n}^{(2)}(v_{n}^{(2)}, 1-p^{2})\cdot(1-I_{p}(n))\\\\
v_{n+1}^{(2)} = v_{n}^{(2)} − B_{n}^{(2)}(v_{n}^{(2)}, 1-p^{2}) + \widetilde{B}_{n}^{(2)}(v_{n}^{(1)} − B_{n}^{(1)}(v_{n}^{(1)}, 1-p), \mu(q_{n})) \\- \widetilde{B}_{n}^{(3)}(v_{n}^{(2)} - B_{n}^{(2)}(v_{n}^{(2)}, 1-p^{2}), \mu(q_{n})) + B_{n}^{(3)}(v_{n}^{(3)}, 1-p^{3})\cdot(1-I_{p}(n))\\
...\\
\end{cases}
\tag{M2'}
\label{Ф2'}
\end{align*}
\normalsize
и доказали, что ее области стабильности и нестабильности совпадают с соответствующими областями модели (\ref{Ф0'}).  Для доказательства использовалась пробная функция
\begin{align}
    L(x) = q + \sum_{i=1}^{\infty}p^{-i/2}v^{(i)}, \quad x = (q, v^{(1)}, v^{(2)},...).
    \label{Lyapunov_function}
\end{align}
Основная идея заключалась в сведении модели (\ref{Ф2'}) к модели (\ref{Ф0'}). При том структура доказательства повторяла структуру доказательства для модели (\ref{Ф0'}). То есть, мы использовали утверждение 2 (Обобщенный критерий Фостера) и утверждение 3 (см. Приложения) для доказательства стабильности и нестабильности соответственно с функцией $g(x)$ следующего вида
\begin{align*}
    g(x) = \begin{cases}
    1,\: L((0, v)) > R\\
    k,\: L((0, v)) \leq R
    \end{cases} \quad v = (v^{(1)}, v^{(2)},...),
\end{align*}
где $L$ --- некоторая, заранее заданная, пробная функция, а значения $R$ и $k$ определяются в процессе доказательства так, что первое значение отделяет в пространстве состояний множество
\begin{align*}
    \{(q, v) \in \mathfrak{X} \: | \: L((0, v)) > R\}, \quad v = (v^{(1)}, v^{(2)},...), \quad \mathfrak{X} \text{ определено в (\ref{State_space})},
\end{align*}
стартуя из которого снос цепи становится отрицательным за один шаг, а второе значение определяет количество шагов $k$ необходимых, чтобы, при достаточно большом значении $q_{0} > \widetilde{q}$, в дополнении к упомянутому множеству, снос цепи становился также отрицательным. Таким образом, в итоге мы имеем некоторое множество 
\begin{align*}
    \{(q, v) \in \mathfrak{X} \: | \: L((0, v)) \leq R,\: q \leq \widetilde{q}\},
\end{align*}
внутри которого снос цепи ограничен, а вне --- отрицателен. Остальные рассуждения были посвящены сведению рассматриваемой цепи (\ref{Ф2'}) к ее двумерному частному случаю (\ref{Ф0'}). Во-первых, мы показали, что снос сообщений, заряженных более, чем на одну ячейку, стремится к нулю с ростом значения $q_{0}$. Во-вторых, мы определили некоторый момент времени, после которого, при достаточно большом значении $q_{0}$, упомянутые цепи совпадают с вероятностью, достаточно близкой к 1, на некотором временном отрезке. Таким образом, при достаточно большом значении $q_{0}$, на некотором временном отрезке, снос цепи (\ref{Ф2'}) определяется сносом цепи (\ref{Ф0'}). 

Проведем аналогичные рассуждения для цепей (\ref{Ф2}) и (\ref{Ф0}). Будем использовать ту же пробную функцию (\ref{Lyapunov_function}). Как было упомянуто, основное отличие системы (\ref{Ф2}) от системы (\ref{Ф2'}) заключается в наличии в первой эффекта саморазрядки. Таким образом, очевидно, что, если системы будут "стартовать"$ $ из одного начального состояния, то будет справедливо следующее соотношение
\begin{align*}
    \sum_{i=1}^{\infty}p^{-i/2}v_{1}^{(i)}(\ref{Ф2}) \leq \sum_{i=1}^{\infty}p^{-i/2}v_{1}^{(i)}(\ref{Ф2'}),
\end{align*}
то есть система с саморазрядкой (\ref{Ф2}) совокупно потеряет больше заряда, чем система без нее (\ref{Ф2'}) за один временной интервал. Следовательно, для сноса цепей за один шаг, будет справедливо следующее соотношение
\begin{align*}
    \E{}_{x}(L(X_{1}) − L(X_{0}))(\ref{Ф2}) \leq \E{}_{x}(L(X_{1}) − L(X_{0}))(\ref{Ф2'})
\end{align*}
То есть, данное соотношение будет сохраняться и для значения $R$
\begin{align}
    R(\ref{Ф2}) \leq R(\ref{Ф2'}),
\end{align}
так как, в силу более интенсивной потери заряда в рассматриваемой модели, нам достаточно не более $R(\ref{Ф2'})$ сообщений, чтобы снос цепи стал отрицательным за один шаг. Аналогичные рассуждения справедливы и для случая нестабильности. Следовательно, мы можем положить $R := R(\ref{Ф2'})$. Таким образом, всюду далее будем считать, что 
\begin{align*}
    L((0, v_{0})) \leq R, \quad v = (v_{0}^{(1)}, v_{0}^{(2)},...).
\end{align*}

Перейдем к сведению цепи (\ref{Ф2}) к (\ref{Ф0}). Проведем рассуждение в два этапа, а именно покажем, что\\
а) снос сообщений, заряженных более чем на одну ячейку, стремится к нулю с ростом значения $q_{0}$,\\
б) при достаточно большом значении $q_{0}$, модели (\ref{Ф2}) и (\ref{Ф0}) совпадают с вероятностью, сколь угодно близкой к 1.\\
$ $
\\
а) Сперва преобразуем рекурсивное выражение, описывающее модель (\ref{Ф2}), заменив $D_{n}^{(i)}(v_{n}^{(i)}, p^{i}(1-\hat{p}))$ на $v_{n}^{(i)} - B_{n}^{(i)}(v_{n}^{(i)}, 1-p^{i}(1-\hat{p}))$, то есть
\begin{align*}
    v_{n+1}^{(i)} = D_{n}^{(i)}(v_{n}^{(i)}, p^{i}(1-\hat{p})) + \widetilde{B}_{n}^{(i)}(D_{n}^{(i-1)}(v_{n}^{(i)}, p^{i}(1-\hat{p}))), \mu(q_{n})) \\- \widetilde{B}_{n}^{(i+1)}(D_{n}^{(i)}(v_{n}^{(i)}, p^{i}(1-\hat{p})), \mu(q_{n})) + D_{n}^{(i+1)}(v_{n}^{(i+1)}, p^{i+1}\hat{p}) \\+ B_{n}^{(i+1)}(v_{n}^{(i+1)}, 1-p^{i+1})\cdot(1-I_{p}(n)) \\= v_{n}^{(i)} - B_{n}^{(i)}(v_{n}^{(i)}, 1-p^{i}(1-\hat{p})) + \widetilde{B}_{n}^{(i)}(D_{n}^{(i-1)}(v_{n}^{(i)}, p^{i}(1-\hat{p}))), \mu(q_{n})) \\- \widetilde{B}_{n}^{(i+1)}(D_{n}^{(i)}(v_{n}^{(i)}, p^{i}(1-\hat{p})), \mu(q_{n})) + D_{n}^{(i+1)}(v_{n}^{(i+1)}, p^{i+1}\hat{p}) \\+ B_{n}^{(i+1)}(v_{n}^{(i+1)}, 1-p^{i+1})\cdot(1-I_{p}(n)),
\end{align*}
где $i \geq 2$ и $n \geq 0$. Также, заметим, что верна следующая оценка
\begin{align}
    v_{n+1}^{(i)} \leq v_{n}^{(i)} - B_{n}^{(i)}(v_{n}^{(i)}, 1-p^{i}(1-\hat{p})) + \widetilde{B}_{n}^{(i)}(D_{n}^{(i-1)}(v_{n}^{(i)}, p^{i}(1-\hat{p}))), \mu(q_{n})) \notag\\- \widetilde{B}_{n}^{(i+1)}(D_{n}^{(i)}(v_{n}^{(i)}, p^{i}(1-\hat{p})), \mu(q_{n})) + D_{n}^{(i+1)}(v_{n}^{(i+1)}, p^{i+1}\hat{p}) \notag\\+ B_{n}^{(i+1)}(v_{n}^{(i+1)}, 1-p^{i+1}),
    \label{Always_collision}
\end{align}
в которой было использовано неравенство $1 - I_{p}(n) \leq 1$.
Рассмотрим снос цепи за $n$ шагов. Сперва воспользуемся неравенством (\ref{Always_collision}) и внесем условное математическое ожидание
\begin{align*}
    \E{}_{x}\Big[\sum_{i=2}^{\infty}p^{-i/2}(v_{n}^{(i)} - v_{0}^{(i)})\Big] \notag\leq \sum_{i=2}^{\infty}p^{-i/2}\Big(v_{n-1}^{(i)} - (1 - p^{i}(1 - \hat{p}))v_{n-1}^{(i)} \\+ p^{i-1}(1-\hat{p})v_{n-1}^{(i-1)}\mu(q_{n-1}) - p^{i}(1-\hat{p})v_{n-1}^{(i)}\mu(q_{n-1}) \\+ p^{i+1}\hat{p}v_{n-1}^{(i+1)} + (1-p^{i+1})v_{n-1}^{(i+1)}\Big)-  \sum_{i=2}^{\infty}p^{-i/2}v_{0}^{(i)}.
\end{align*}
С помощью простых преобразований получим следующее выражение
\begin{align*}
    \E{}_{x}\Big[\sum_{i=2}^{\infty}p^{-i/2}(v_{n-1}^{(i)} - v_{0}^{(i)}) \\- \sum_{i=2}^{\infty}p^{-i/2}(1 - p^{i})v_{n-1}^{(i)} + \sum_{i=2}^{\infty}p^{-i/2}(1 - p^{i+1})v_{n-1}^{(i+1)} \\- \sum_{i=2}^{\infty}p^{i/2}\hat{p}v_{n-1}^{(i)} + \sum_{i=2}^{\infty}p^{i/2+1}\hat{p}v_{n-1}^{(i+1)} \\+ \sum_{i=2}^{\infty}p^{i/2-1}(1-\hat{p})v_{n-1}^{(i-1)}\mu(q_{n-1}) - \sum_{i=2}^{\infty}p^{i/2}(1-\hat{p})v_{n-1}^{(i)}\mu(q_{n-1})\Big].
\end{align*}
Тогда ясно, что выражение сверху не превышает
\begin{align*}
    \E{}_{x}\Big[\sum_{i=2}^{\infty}p^{-i/2}(v_{n-1}^{(i)} - v_{0}^{(i)})\Big] + \sum_{i=1}^{\infty}\E{}_{x}v_{n-1}^{(i)}\mu(q_{n-1}).
\end{align*}
Продолжая неравенства далее таким же образом, в итоге получим следующую оценку
\begin{align}
    \E{}_{x}\Big[\sum_{i=2}^{\infty}p^{-i/2}(v_{n}^{(i)} - v_{0}^{(i)})\Big] \leq \E{}_{x}\sum_{l=0}^{n-1}\mu(q_{l})\sum_{i=1}^{\infty}v_{l}^{(i)}.
    \label{Drift_inequality_i2}
\end{align}
Заметим, что за один временной интервал, скажем $[l-1, l)$, общее количество заряженных сообщений в среднем вырастет не более, чем на $\widetilde{c} + \lambda\mu(q_{l})$. Тогда, за $n$ шагов, в силу того, что в начальном состоянии цепи, заряженных сообщений не более, чем $R$, значит общее количество заряженных сообщений не превысит
\begin{align*}
    R + (\widetilde{c} + \lambda\mu(q_{0})) + (\widetilde{c} + \lambda\mu(q_{1})) + ... + (\widetilde{c} + \lambda\mu(q_{n-1})).
\end{align*}
Также, так как за один временной интервал из системы может уйти не более одного сообщения, следовательно $q_{l} \geq q_{l-1}-1$ п.н. Другими словами, справедлива оценка $\mu(q_{l}) \leq \widetilde{c}/(q_{0}-n+1)$, при $l \leq n-1$ и $q_{0} \geq n$. Таким образом, неравенство (\ref{Drift_inequality_i2}) не превосходит
\begin{align*}
    \frac{\widetilde{c}n\Big(R + n\big(\widetilde{c} + \lambda\mu(q_{l})\big)\Big)}{q_{0}-n+1}.
\end{align*}
Отсюда ясно, что при росте значения $q_{0}$, снос сообщений, заряженных на более, чем одну ячейку стремится к нулю.
\\
$ $
\\
б) Введем два множества событий. Пусть
\begin{gather*}
    A_{0}(n) = \{\text{после ($n-1$)-ого шага все изначально заряженные сообщения}\\ \text{потеряли весь свой заряд или покинули систему}\},\\
    A_{2}(q_{0}, k) = \{\widetilde{B}^{(i)}_{n}(v^{(i-1)}_{n} - B^{(i-1)}_{n}(v^{(i-1)}_{n}, 1 - p^{i-1}), \mu(q_{n})) = 0 \\\:\:\text{для всех} \:\: 2 \leq i < \infty \:\: \text{и} \:\: 0 \leq n \leq k\}.
\end{gather*}
Можно заметить, что, при некоторых $k$ и $n_{0} \leq k$, внутри события $A_{0}(n_{0}) \cap A_{2}(q_{0}, k)$ цепи (\ref{Ф2}) и (\ref{Ф0}) совпадают на временном отрезке $[n_{0}, k]$, так как, во-первых, начальные сообщения после момента времени $n_{0}$ потратили весь свой заряд и, во-вторых, за $k$ шагов не было заряжено на более чем одно деление ни одного сообщения. Следовательно, с момента времени $n_{0}$ по момент времени $k$ в системе не было сообщений, заряженных более, чем на одно деление. Таким образом, для доказательства данного пункта, необходимо показать, что
\begin{align}
    \Pb{}_{x}(A_{0}(n_{0}) \cap A_{2}(q_{0}, k)) \xrightarrow{} 1, \text{ при } q_{0} \xrightarrow{} \infty
    \label{Merging_chains}
\end{align}
Сначала покажем, что сообщения, заряженные на одну ячейку $\{v^{(1)}_{n}\}_{n \geq 0}$ --- собственные случайные величины. Для этого достаточно заметить, что, за один временной слот, скажем $[n-1, n)$, условное математическое ожидание количества заряженных сообщений не прирастает больше, чем на $\widetilde{c} + \lambda\mu(q_{n})$, а также в начальном состоянии цепи (\ref{Ф2}) количество заряженных сообщений не превосходит $R$, так как ясно, что $\norm{v} \leq L((0, v))$, при всех значениях $p \in (0, 1)$. Таким образом, верна следующая оценка
\begin{align*}
    \E{}_{x}v_{n}^{(1)} \leq n\big(\widetilde{c} + \lambda\big)
    \label{Expectation_vn_simple_estimation}
\end{align*}
Следовательно, случайные величины $\{v_{n}^{(1)}\}_{n \geq 0}$ --- собственные. Тогда ясно, что вероятность события $A_{1}(C, k) = \{v_{n}^{(1)} \leq C, \text{ при } 0 \leq n \leq k\}$, при любом фиксированном значении $k$, путем выбора достаточно большого значения $C$, может быть сколь угодно близкой к 1. Выберем некоторое произвольное значение $0 < \delta < 1$ и параметры $k$, $C$ такие, что
\begin{align*}
    \Pb{}_{x}(A_{1}(C, k)) \geq 1 - \delta/2.
\end{align*}
Заметим, что верна следующая цепочка неравенств
\begin{align*}
    \Pb{}_{x}(A_{2}(q_{0}, k) \cap A_{1}(C, k)) \geq \Pb{}_{x}(A_{2}(q_{0}, k) \: | \: v_{1}^{(1)} \leq C, v_{2}^{(1)} \leq C,..., v_{k}^{(1)} \leq C)(1 - \delta/2) \\\geq \Pb{}_{x}(\{\widetilde{B}^{(1)}_{n}(R + C, \mu(q_{n})) = 0 \:\: \text{для всех} \:\: 0 \leq n \leq k \})(1 - \delta/2), 
\end{align*}
где последнее неравенство справедливо в силу того, что в начальном состоянии $(L((0, v_{0})) \leq R)$ не более чем $R$ заряженных сообщений, а также вероятность подзарядки одного сообщения в произвольный момент времени не зависит от того, какой заряд оно имеет. Ясно, что выражение сверху не превосходит
\begin{align*}
    (1 - \mu(q_{0} - k))^{(R + C)k}(1 - \delta/2).
\end{align*}
Таким образом, путем выбора достаточно большого значения $q_{0}$, получим
\begin{align*}
    \Pb{}_{x}(A_{2}(q_{0}, k) \cap A_{1}(C, k)) \geq 1 - \delta/2.
\end{align*}
Осталось заметить, что, при некотором достаточно большом значении $n_{0}$, вероятность события $A_{0}(n_{0})$ будет не меньше, чем $1 - \delta/2$, так как ясно, что за достаточно большой промежуток времени ограниченное количество заряда израсходуется с вероятностью, сколь угодно близкой к 1. Тогда очевидно выполняется следующая цепочка неравенств
\begin{align*}
    \Pb{}_{x}(A_{0}(n_{0}) \cap A_{2}(q_{0}, k)) \geq \Pb{}_{x}(A_{0}(n_{0}) \cap A_{1}(C, k) \cap A_{2}(q_{0}, k)) - \delta/2 \geq 1 - \delta.
\end{align*}
Таким образом, в силу произвольности значения $\delta$, выражение (\ref{Merging_chains}) выполнено.


\section{Замечание об ограниченной вместимости}
Представляет интерес модель, в которой предполагается наличие батарейки лишь с ограниченным числом ячеек для хранения энергии у каждого пользователя. При этом модель (\ref{Ф2}), в случае наличия батарейки с $m$ ячейками, будет выглядить следующим образом:
\small
\begin{equation*}
\begin{cases}
q_{n+1} = q_{n} − I_{p}(n) + \xi_{n}, \\\\
v_{n+1}^{(1)} = D_{n}^{(1)}(v_{n}^{(1)}, p(1-\hat{p})) + \widetilde{B}_{n}^{(1)}(q_{n} − \sum_{i=1}^{\infty}v_{n}^{(i)} + \xi_{n}, \mu(q_{n})) \\- \widetilde{B}_{n}^{(2)}(D_{n}^{(1)}(v_{n}^{(1)}, p(1-\hat{p})), \mu(q_{n})) + D_{n}^{(2)}(v_{n}^{(2)}, p^{2}\hat{p}) \\+ B_{n}^{(2)}(v_{n}^{(2)}, 1-p^{2})\cdot(1-I_{p}(n))\\\\
v_{n+1}^{(2)} = D_{n}^{(2)}(v_{n}^{(2)}, p^{2}(1-\hat{p})) + \widetilde{B}_{n}^{(2)}(D_{n}^{(1)}(v_{n}^{(1)}, p(1-\hat{p}))), \mu(q_{n})) \\- \widetilde{B}_{n}^{(3)}(D_{n}^{(2)}(v_{n}^{(2)}, p^{2}(1-\hat{p})), \mu(q_{n})) + D_{n}^{(3)}(v_{n}^{(3)}, p^{3}\hat{p}) \\+ B_{n}^{(3)}(v_{n}^{(3)}, 1-p^{3})\cdot(1-I_{p}(n))\\
...\\
v_{n+1}^{(m)} = D_{n}^{(m)}(v_{n}^{(m)}, p^{m}(1-\hat{p})) + \widetilde{B}_{n}^{(m)}(D_{n}^{(m-1)}(v_{n}^{(m-1)}, p^{m-1}(1-\hat{p}))), \mu(q_{n}))
\end{cases}
\tag{Mm}
\label{Фm}
\end{equation*}
\normalsize
Несложно понять, что условия стабильности и нестабильности у данной модели не изменятся. Действительно, аналогично рассуждениям в доказательстве для цепи (\ref{Ф2}), выберем пробную функцию (\ref{Lyapunov_function}) и покажем, что цепь (\ref{Фm}) сводится к цепи (\ref{Ф2}). Для этого достаточно "запустить"$ $ упомянутые цепи из одного начального состояния и заметить, что из пункта (а) доказательства основной теоремы, снос сообщений, заряженных более, чем на $m$ ячеек, стремится к 0 с ростом значения $q_{0}$ и из пункта (б) вероятность подзарядки заряженных сообщений, путем выбора достаточно большого значения $q_{0}$, можно сделать сколь угодно близкой к 0. Таким образом, цепи будут совпадать с вероятностью, близкой к 1, и снос цепи (\ref{Фm}) будет определяться сносом цепи (\ref{Ф2}).\\

\section{Заключение} 
Мы исследовали стабильность и нестабильность системы случайного множественного доступа, управлемой протоколом ALOHA, с бесконечным числом пользователей, дополнительно снабженных механизмом энергетической подпитки с аккумуляторами бесконечной емкости. Также, мы предположили, что аккумуляторы подвержены эффекту саморазрядки, то есть потери заряда в период простоя. Были получены условия стабильности и нестабильности системы.

Главное отличие от результатов, полученных в работе \cite{Rezler_Chebunin}, где была исследована аналогичная модель, однако не учитывающая возможность саморазрядки, состоит в изменении пропускной способности системы, при заданной интенсивности подзарядки. При том, максимальная пропускная способность остается прежней.

Авторы выражают благодарность за постановку задачи, полезные обсуждения и замечания в ходе работы над этой статьей Фоссу С.Г.

\section{Приложения}
Сформулируем вспомогательные утверждения, используемые при доказательстве основной теоремы. Сначала приведем лемму 1 из работы \cite{Foss_original}. Мы будем также далее использовать следующие обозначения $D_{n}(k, p) := k - B_{n}(k, 1-p) = \sum_{i=1}^{k}I(U_{n,i} > 1-p)$, $\widetilde{D}_{n}(k, p) := k - \widetilde{B}_{n}(k, 1-p) = \sum_{i=1}^{k}I(\widetilde{U}_{n,i} > 1-p)$. Ясно, что $D_{n}(k, p)$ и $\widetilde{D}_{n}(k, p)$ имеют биномиальное распределение с параметрами $k$ и $p$.

\begin{numbered_theorem}
Пусть $\{Z_{n}\}_{n \geq 0}$ последовательность н.о.р. неотрицательных целочисленных случайных
величин с конечным средним $\E{}Z_{0} < \infty$. Предположим, что они не зависят от семейств н.о.р. случайных величин \{$U^{(j)}_{n,i}$; $-\infty$ < n < $\infty$, $i \geq 1$\} и \{$\widetilde{U}^{(j)}_{n,i}$; $-\infty$ < n < $\infty$, \:$i \geq 1, \:1 \leq j \leq m$\}, имеющих равномерное распределение на интервале (0, 1).
\\
\\
(I) Для любого начального значения $W_{0}$ и для любого параметра $p \in (0, 1)$, последовательность
\\
\begin{equation}
    W_{n+1} = W_{n} − B_{n}(W_{n}, 1-p) + Z_{n} \equiv D_{n}(W_{n}, p) + Z_{n}
\end{equation}
\quad эргодична.
\\\\
(II) Для $m \leq n$ определим случайные величины
\begin{equation}
    D_{m:n}(k, p) = D_{n}(D_{n−1}(...(D_{m+1}(D_{m}(k, p), p)..., p)
\end{equation}
и заметим, что случайная величина $D_{m:n}(k, 1 − p)$ имеет биномиальное распределение с
параметрами $k$ и $p^{n−m+1}$. Тогда стационарная последовательность {$W^{(n)}$}, определенная следующим образом
\begin{equation}
    W^{(n)} = Z_{n−1} + \sum_{j=1}^{+\infty} D_{(n−j):(n−1)}(Z_{n−j−1}, p),
    \label{stationary_solution_of_rec_eq}
\end{equation}
имеет конечное математическое ожидание $EW^{(n)} = \frac{EZ_{0}}{1-p}$ и образует стационарное решение для рекурсивного уравнения,
\begin{equation}
    W^{(n+1)} = D_{n}(W^{(n)}, p) + Z_{n}.
    \label{recursive_equation}
\end{equation}
(III) Пусть Q --- распределение $W^{(0)}$. Тогда, для любого начального значения $W_{0}$
\begin{equation}
    \Pb(W_{n} = W^{(n)}) = \Pb(W_{l} = W^{(l)}, \forall l \geq n) \xrightarrow[]{} 1
\end{equation}
и, в частности,
\begin{equation}
    \underset{\rm A \in Z_{+}}{\rm sup}|\Pb(W_{n} \in A) − Q(A)| \leq \Pb(W_{n} \neq W^{(n)}) \xrightarrow{} 0, \:\: \text{при} \:\: n \xrightarrow{} \infty.
\end{equation}
В частности, если $W_{0}$ и $Z_{0}$ имеют конечные экспоненциальные моменты, тогда
\begin{equation}
    \Pb(W_{n} \neq W^{(n)}) \leq K_{1}e^{−K_{2}n},
\end{equation}
для некоторых $K_{1}, K_{2} > 0$ и для всех $N \geq 0$.
\\
(IV) Для любого $n \geq 0$,
\begin{equation}
    W_{n} \leq W_{0} + W^{(n)} \:\: \text{п.н.}
\end{equation}
и
\begin{equation}
    \E{}W_{n} \leq \E{}W_{0} + \frac{\E{}Z_{0}}{1-p}.
\end{equation}
(V) Пусть $\widetilde{Z}_{n}$ --- любая другая последовательность неотрицательных целочисленных случайных величин, таких что $0 \leq \widetilde{Z}_{n} \leq Z_{n}$ п.н., для всех $n$. Рассмотрим рекурсивное уравнение
\begin{equation}
    \widetilde{W}_{n+1} = D_{n}(\widetilde{W}_{n}, p) + \widetilde{Z}_{n}
\end{equation}
с целочисленным начальным значением $0 \leq \widetilde{W}_{0} \leq W_{0}$. Тогда
\begin{equation}
    \widetilde{W}_{n} \leq W_{n} \quad \text{п.н.}, \:\: \text{для всех} \:\: n \geq 0.
\end{equation}
(VI) В частности, если $Z_{0}$ пуассоновская случайная величина с параметром $c$, то каждый $W^{(n)}$ имеет пуассоновское распределение с параметром $c/(1-p)$.
\end{numbered_theorem}
Далее сформулируем второе вспомогательное утверждение из работы \text{\cite{Foss_Konstantinopolous}}. Пусть $X_{n}$ --- однородная по времени цепь Маркова, принимающая значения в некотором польском пространстве $\mathcal{X}$. Пусть $L:\mathcal{X} \to \mathbb{R}_{+}$ --- некоторая измеримая функция Ляпунова. Пусть $g:\mathcal{X} \to \mathbb{N}$ и $h:\mathcal{X} \to \mathbb{R}$ --- измеримые функции такие, что:\\
1) $h$ ограничена снизу: $\underset{x \in \mathcal{X}}{h(x)} > -\infty$;\\
2) $h$ <<в конце>> положительна: $\underset{K > 0}{\sup}$ $\underset{L(x) > K}{\inf}h(x) > 0$;\\
3) $g$ локально ограничена сверху: $\underset{L(x) \leq N}{\sup}g(x) < \infty$ для всех $N \geq 0$;\\
4) $g$ <<в конце>> ограничена функцией $h$: $\underset{K > 0}{\inf}$ $\underset{L(x) > K}{\sup}\frac{g(x)}{h(x)} < \infty$.\\
Для измеримого множества $B \subseteq \mathcal{X}$ определим $\tau_{B} = \inf\{n \geq 1: X_{n} \in B\}$. Множество $B$ называется \textit{возвратным}, если $\Pb{}_{x}(\tau_{B} < \infty) = 1$ и \textit{положительно возвратным}, если $\underset{x \in \mathcal{X}}{\sup}\E{}_{x}\tau_{B} < \infty$. Теперь мы готовы сформулировать утверждение.
\begin{numbered_theorem}
(обобщенный критерий Фостера)
Предположим, что существуют тестовая функция (<<функция Ляпунова>>) $L(x) \geq 0$, и функции $g$ и $h$, удовлетворяющие условиям $1)-4)$ такие, что
\begin{equation}
    \E(L(X_{g(X_{0})}) − L(X_{0}) | X_{0}=x) \leq -h(x).
    \label{foster_property_2_app}
\end{equation}
Тогда существует $N_{0} \in \mathbb{N}$ такая, что для всех $N > N_{0}$ и всех $x \in \mathcal{X}$ имеем $\E{}_{x}\tau < \infty$ и $\underset{L(x) \leq N}{\E{}_{x}\tau} < \infty$, где $\tau \equiv \tau(N) = \inf\{n \geq 1 : L(x) \leq N\}$.
\end{numbered_theorem} 

${}$\\ Сформулируем третье вспомогательное утверждение, которое было исследовано в работе С.Г. Фосса и Д.Е. Денисова \cite{Foss_theorem_for_instability}. Данное утверждение более аккуратно изложено, а также доказано при более общих условиях в работе Д.Е. Денисова [?]. Но ради простоты мы приведем только частный случай для однородных цепей Маркова. Пусть последовательность $\{X_{n}\}_{n \geq 0}$ образует цепь Маркова такую, что
\begin{equation*}
    X_{n+1} = f(X_{n}, \alpha_{n})
\end{equation*}
с начальным значением $X_{0} = x$, где $\{\alpha_{n}\}_{n \geq 0}$ образует последовательность независимых равномерно распределенных на отрезке $[0, 1]$ случайных величин, а $f$ --- измеримая функция. Пусть далее $\Delta = L(X_{1}) - L(X_{0})$, где $L$ --- измеримая функция. Определим
\begin{equation*}
    \tau(N) = \inf\{n \geq 0\: : \: L(X_{n}) \geq N \}.
\end{equation*}
Также далее полагаем, что $h(t)$ --- некоторая вещественная функция такая, что $(h(t))^{-1}$ интегрируема на промежутке $(1, \infty)$.\\
\begin{numbered_theorem}
Пусть существуют такие числа $N > 0$, $\varepsilon > 0$ и такая функция $h(t)$, что\\
1) $\tau(N) < \infty$ п.н. для произвольного начального состояния цепи;\\ 
2) Для всех $x \in X$ таких, что $L(x) \geq N$, выполняется:
\begin{equation*}
    \E{}_{x}(\Delta \cdot I(\Delta \leq M)) \geq \varepsilon;
\end{equation*}
3) Семейство случайных величин $\{h(-min(0, \Delta)),\: X_{0}=x, \:L(x) \geq N\}$ равномерно интегрируемо. Тогда для каждого $x \in X$ \\
\begin{equation*}
    \Pb{}_{x}(\lim_{n\to\infty}L(X_{n}) = \infty) = 1.
\end{equation*}
\end{numbered_theorem}

\begin{thebibliography}{1}
\bigskip
\footnotesize
\bibitem{Foss_original}
S. Foss, D. Kim, A. Turlikov, Stability and instability of a random multiple access model with adaptive energy harvesting, Siberian
Electronic Mathematical Reports, 13 (2016), 16–25. MR3506879
\bibitem{Rezler_Chebunin}
А. В. Резлер, М. Г. Чебунин, “Стабильность и нестабильность систем случайного множественного доступа с механизмом энергетической подпитки”, Сиб. электрон. матем. изв., 19:1 (2022), 1–17
\bibitem{Harb_Adnan}
Harb, Adnan. (2011). Energy harvesting: State-of-the-art. Renewable Energy. 36. 2641-2654. 10.1016/j.renene.2010.06.014. 
\bibitem{Sudevalayam_Kulkarni}
S. Sudevalayam and P. Kulkarni, "Energy Harvesting Sensor Nodes: Survey and Implications," in IEEE Communications Surveys & Tutorials, vol. 13, no. 3, pp. 443-461, Third Quarter 2011, doi: 10.1109/SURV.2011.060710.00094.
\bibitem{Example_of_applying_Xun_Zhou}
Xun Zhou, Rui Zhang, Chin Keong Ho, Wireless Information and Power Transfer in Multiuser OFDM Systems, IEEE Transactions on Wireless Communications, 13:4 (2014), 2282–2294.
\bibitem{Jeon_finite_users}
J. Jeon and A. Ephremides, The stability region of random multiple access under stochastic energy harvesting, Proceedings of the IEEE International Symposium on Information Theory (ISIT), (2011), 1796–1800.
\bibitem{Efficiency_protocols}
A. Bergman, M. Sidi, Energy efficiency of collision resolution protocols, Computer Communications, 29 (2006), 3397–3415.
\bibitem{Abramson_dev_of_aloha_net}
N. Abramson, Development of the ALOHANET, IEEE Trans. Info. Theory, 31 (1985), 119– 123. Zbl 0563.94001
\bibitem{FGL_1977}
G. Fayolle, E. Gelenbe, J. Labetoulle, Stability and optimal control of the packet switching broadcast channel, Journal of the ACM (JACM), 24:3 (1977), 375–386. MR0445639
\bibitem{Kelly_classical_aloha_instability}
J. P. Kelly and I. M. McPhee, The Number of Packets Transmitted by Collision Detect Random Access Schemes, Annals of Probability, 15:4 (1987), 1557–1568. MR0905348
\bibitem{Vvedenskaya_metastability}
N. Vvedenskaya, Yu. Suhov, Multi-access system with many users: Stability and metastability, Problems of Information Transmission, 43:3 (2007), 263–269. MR2360021
\bibitem{Foss_2014}
D. Kim, A. Turlikov, S. Foss, Random multiple access with common energy harvesting mechanism, Siberian
Electronic Mathematical Reports, 11 (2014), 896–905. MR3488468
\bibitem{Foss_theorem_for_instability}
Foss S. G. and Denisov D. E., On transience conditions for Markov chains, Siberian Math. J., 42, No. 2, 364–371 
(2001). MR183316
\bibitem{Foss_Konstantinopolous}
S. Foss and T. Konstantopoulos, An overview of some stochastic stability methods, Journal of Operation Research Society Japan, 47:4 (2004), 275–303. Zbl 1134.93412

\end{thebibliography}
\end{document}



